

\subsubsection{Features based on codon analysis}
This category measures the bias in codon usage relative  to endogenous mRNAs. Higher values of these indices is an indicator that the given mRNA sequence follows the codon usage pattern of the host. Features under this category are the codon adaptation index (CAI) \cite{Sharp1987-ed}, tRNA adaptation index (tAI) \cite{ Reis2004-dl, Sabi2014-je} and related metrics such as codon pair usage \cite{Gutman1989-pn}. For example: the codon adaptation index (CAI) for a given protein is the harmonic mean of the relative adaptiveness $w$ \cite{Sharp1987-ed} of the codons:

\begin{equation}
    CAI_{g}=(\prod_{i=1}^{N} w_i)^{1/N}
    \label{eqn:cai}
\end{equation}
where $w_i$ is the relative adaptiveness of the $i^{th}$ codon which is the ratio of observed frequency of the codon $f_i$ upon consideration to the frequency of the most frequent synonymous codon. $$w_i = \frac{f_i}{max(f_i)}$$ 

Based upon the idea of CAI, tAI was developed to measure the translational efficiency by taking into account of tRNA concentration and codon-anticodon coupling efficiency. We first define the absolute adaptiveness $W_i$ of codon $i$ as:

\begin{equation}
    W_i = \sum_{j=1}^{n_i} (1 - s_{ij})tGCN_{ij}
\end{equation}
where $n_i$ is the number of anticodons pairing with codon $i$, $tGCN_{ij}$ is the copy number of the $j^{th}$ tRNA that recognizes the $i^{th}$ codon. $tGCN_{ij}$ is correlated with tRNA concentration \cite{kanaya1999studies, }. $s_{ij}$ is a constraint on the codon-anticodon pairing and has values between $0$ (more efficient pairing) and $1$ (less efficient pairing). The relative adaptiveness $w_i$ of the $i^{th}$ codon is  $W_i$ normalised by maximum of $W_i$ among all codons. If $W_i$ is zero, then the relative adaptiveness is the mean of all $W_i$. Once $w_i$ are found, the tAI is the harmonic mean as in Equation \ref{eqn:cai}. 


Both of these measures are equivalent to a zeroth order Markov model whereas codon pair usage or di-codon frequency is essentially a first order Markov model. It is thought that a higher value of these indices means that the sequence can utilise the available tRNA pool more efficiently which causes an increase in efficiency of translation \cite{ikemura1985codon, Gutman1989-pn, Sharp1987-ed, Reis2004-dl, Sabi2014-je, Brule2017-mx}. However, this proposition has been challenged and studies suggest that mRNA secondary structure might be more important in explaining translation efficiency.  \cite{Kudla2009-tl, Boel2016-jd, Cambray2018-kn}.