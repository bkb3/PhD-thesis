\begin{abstract}
%Here is the abstract for this thesis.
Recombinant protein production is a cornerstone of modern biotechnology and has been utilised to produce many proteins of scientific and commercial interest since around the last four decades. The fundamental idea is the expression of a recombinant gene in an expression host. Hence, the optimality of the result is dependent on the balances among the involved intricate stochastic processes. In particular, two of the critical processes are protein expression and solubility. Collectively, the failures at these two steps drop down the success rate of protein production to around $25\%$. Furthermore, toxicity of recombinant proteins may also significantly reduce the amount of protein produced. Therefore, prediction and optimisation of expression, solubility and an early detection of these toxic proteins could save resources and assist in better planning of the experiment.

In this work, we show that mRNA accessibility, measured through the opening energy, and protein structural flexibility, measured by using the normalised B-factors, can describe protein expression and solubility respectively with a higher accuracy than other features. We also develop a new and more accurate protein solubility predicting metric called the Solubility-Weighted Index (SWI). Using these findings, we develop a gene expression prediction and optimisation tool: Translation Initiation coding region designer (TIsigner), available at \url{https://tisigner.com/tisigner} and protein solubility prediction and optimisation tool: Soluble Domain of Protein Expression (SoDoPE), available at \url{https://tisigner.com/sodope}. We also developed a third tool, Razor \url{https://tisigner.com/razor}, for the detection of toxins. To assist in maximising protein production, we also develop a pipeline for optimising protein expression, solubility and toxin detection by integrating these three tools.
 
%The first bottleneck in protein production is mRNA expression. This problem is quite intriguing because previous studies show that mRNA concentration can explain just around $40\%$ of the variation in protein yield. Several mRNA features have been proposed to explain this discrepancy, but the accuracy of predicting protein expression using these features is still low. Furthermore, many of the proposed features are not independent, which makes it hard to distinguish the impacts of individual features. After successful expression, obtaining soluble protein is the next step towards many structural and functional studies. However, nearly half of the expressed proteins remain insoluble. Prediction of protein solubility is a complicated task as it depends upon the three dimensional structure of protein and interactions with the solvent. Nevertheless, useful approximations to the behaviour of protein can be inferred by using the features of the primary sequence. 
% Recombinant protein production is a cornerstone of modern biotechnology. Although the technology has a more than four decades long history, many experiments still produce sub-optimal results or even fail. Previous studies show that mRNA concentration can explain just around 40\% of the variation in protein yield. Several mRNA features have been proposed to explain this discrepancy, but the accuracy of predicting protein expression using these features is still low. Many of the proposed features are not independent, which makes it hard to distinguish the impacts of individual features.


% After successful expression, obtaining soluble protein is the next step towards many structural and functional studies. However, nearly half of the expressed protein remain insoluble, hence protein solubility still is a bottleneck for protein production. A number of methods have been proposed to change the intrinsic features of a protein and consequently improve its solubility. A prediction of protein solubility before the experiment could save resources and assist in better planning of the experiment. 


% A wide variety of tools exist in the area of both optimising protein expression and solubility. Many of these tools either depend upon older data or take long time to execute. Surprisingly, none of these tools offer the pipeline for improving expression and solubility at the same time. This pipeline is crucial to get the best results form an experiment because high levels of protein expression followed by high solubility is the logical step in protein expression. Thus, the tools to design genes with a precisely predicted protein yield is lacking. 


% In this study, we show that mRNA accessibility, measured through the opening energy, and protein structural flexibility, measured by using the normalised B-factors, can describe protein expression and solubility respectively with a higher accuracy than other features. We also develop a new and more accurate protein solubility predicting metric called the Solubility Weighted Index (SWI). Using these findings, we develop a gene expression prediction and optimisation tool: Translation Initiation coding region designer (TIsigner), available at \url{https://tisigner.com/tisigner} and protein solubility prediction and optimisation tool: Soluble Domain of Protein Expression (SoDoPE), available at \url{https://tisigner.com/sodope}. To assist in maximising protein production, we also develop a pipeline for optimising both protein expression and solubility by integrating these two tools.


% can predict protein solubility We then use this feature to develop a tool (Translation Initiation coding region designer (TIsigner)) which can be used to fine tune the protein expression to any desired level from low to high. 

% These features can be used for prediction and optimisation of protein expression and solubility. A suitable pipeline which integrates these optimisation tools can then be developed. 



\end{abstract}